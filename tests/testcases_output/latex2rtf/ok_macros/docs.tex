\documentclass{report}
\usepackage{hyperref}
% WARNING: THIS SHOULD BE MODIFIED DEPENDING ON THE LETTER/A4 SIZE
\oddsidemargin 0cm
\evensidemargin 0cm
\marginparsep 0cm
\marginparwidth 0cm
\parindent 0cm
\textwidth 16.5cm

\ifpdf
  \usepackage[pdftex]{graphicx}
\else
  \usepackage[dvips]{graphicx}
\fi

\begin{document}
% special variable used for calculating some widths.
\newlength{\tmplength}
\chapter{Unit ok{\_}macros}
\section{Description}
Test of FPC macros handling.\hfill\vspace*{1ex}



Parts based on [http://sourceforge.net/tracker/index.php?func=detail{\&}aid=861356{\&}group{\_}id=4213{\&}atid=354213]
\section{Overview}
\begin{description}
\item[\texttt{\begin{ttfamily}TAncestor\end{ttfamily} Class}]
\item[\texttt{\begin{ttfamily}TMyClass\end{ttfamily} Class}]
\end{description}
\begin{description}
\item[\texttt{MyProc1}]
\item[\texttt{MyProc2}]
\item[\texttt{Foo}]
\item[\texttt{MyProc3}]
\item[\texttt{ThisShouldBeIncluded}]
\item[\texttt{ThisShouldBeIncluded2}]
\end{description}
\section{Classes, Interfaces, Objects and Records}
\subsection*{TAncestor Class}
\subsubsection*{\large{\textbf{Hierarchy}}\normalsize\hspace{1ex}\hfill}
TAncestor {$>$} TObject
%%%%Description
\subsection*{TMyClass Class}
\subsubsection*{\large{\textbf{Hierarchy}}\normalsize\hspace{1ex}\hfill}
TMyClass {$>$} \begin{ttfamily}TAncestor\end{ttfamily}(\ref{ok_macros.TAncestor}) {$>$} 
TObject
%%%%Description
\subsubsection*{\large{\textbf{Methods}}\normalsize\hspace{1ex}\hfill}
\paragraph*{Init}\hspace*{\fill}

\begin{list}{}{
\settowidth{\tmplength}{\textbf{Description}}
\setlength{\itemindent}{0cm}
\setlength{\listparindent}{0cm}
\setlength{\leftmargin}{\evensidemargin}
\addtolength{\leftmargin}{\tmplength}
\settowidth{\labelsep}{X}
\addtolength{\leftmargin}{\labelsep}
\setlength{\labelwidth}{\tmplength}
}
\begin{flushleft}
\item[\textbf{Declaration}\hfill]
\begin{ttfamily}
public Constructor Init; Overload;\end{ttfamily}


\end{flushleft}
\end{list}
\section{Functions and Procedures}
\subsection*{MyProc1}
\begin{list}{}{
\settowidth{\tmplength}{\textbf{Description}}
\setlength{\itemindent}{0cm}
\setlength{\listparindent}{0cm}
\setlength{\leftmargin}{\evensidemargin}
\addtolength{\leftmargin}{\tmplength}
\settowidth{\labelsep}{X}
\addtolength{\leftmargin}{\labelsep}
\setlength{\labelwidth}{\tmplength}
}
\begin{flushleft}
\item[\textbf{Declaration}\hfill]
\begin{ttfamily}
procedure MyProc1( a:Integer);\end{ttfamily}


\end{flushleft}
\par
\item[\textbf{Description}]
Below is an example of a very bad and confusing (but valid) macro usage. Just to test pasdoc.

\end{list}
\subsection*{MyProc2}
\begin{list}{}{
\settowidth{\tmplength}{\textbf{Description}}
\setlength{\itemindent}{0cm}
\setlength{\listparindent}{0cm}
\setlength{\leftmargin}{\evensidemargin}
\addtolength{\leftmargin}{\tmplength}
\settowidth{\labelsep}{X}
\addtolength{\leftmargin}{\labelsep}
\setlength{\labelwidth}{\tmplength}
}
\begin{flushleft}
\item[\textbf{Declaration}\hfill]
\begin{ttfamily}
procedure MyProc2( b: Integer);\end{ttfamily}


\end{flushleft}
\par
\item[\textbf{Description}]
This is very stupid way to declare a procedure

\end{list}
\subsection*{Foo}
\begin{list}{}{
\settowidth{\tmplength}{\textbf{Description}}
\setlength{\itemindent}{0cm}
\setlength{\listparindent}{0cm}
\setlength{\leftmargin}{\evensidemargin}
\addtolength{\leftmargin}{\tmplength}
\settowidth{\labelsep}{X}
\addtolength{\leftmargin}{\labelsep}
\setlength{\labelwidth}{\tmplength}
}
\begin{flushleft}
\item[\textbf{Declaration}\hfill]
\begin{ttfamily}
function Foo(c: string): Integer;\end{ttfamily}


\end{flushleft}
\end{list}
\subsection*{MyProc3}
\begin{list}{}{
\settowidth{\tmplength}{\textbf{Description}}
\setlength{\itemindent}{0cm}
\setlength{\listparindent}{0cm}
\setlength{\leftmargin}{\evensidemargin}
\addtolength{\leftmargin}{\tmplength}
\settowidth{\labelsep}{X}
\addtolength{\leftmargin}{\labelsep}
\setlength{\labelwidth}{\tmplength}
}
\begin{flushleft}
\item[\textbf{Declaration}\hfill]
\begin{ttfamily}
procedure MyProc3( X: Integer = 1; Y: Integer = 2);\end{ttfamily}


\end{flushleft}
\end{list}
\subsection*{ThisShouldBeIncluded}
\begin{list}{}{
\settowidth{\tmplength}{\textbf{Description}}
\setlength{\itemindent}{0cm}
\setlength{\listparindent}{0cm}
\setlength{\leftmargin}{\evensidemargin}
\addtolength{\leftmargin}{\tmplength}
\settowidth{\labelsep}{X}
\addtolength{\leftmargin}{\labelsep}
\setlength{\labelwidth}{\tmplength}
}
\begin{flushleft}
\item[\textbf{Declaration}\hfill]
\begin{ttfamily}
procedure ThisShouldBeIncluded;\end{ttfamily}


\end{flushleft}
\end{list}
\subsection*{ThisShouldBeIncluded2}
\begin{list}{}{
\settowidth{\tmplength}{\textbf{Description}}
\setlength{\itemindent}{0cm}
\setlength{\listparindent}{0cm}
\setlength{\leftmargin}{\evensidemargin}
\addtolength{\leftmargin}{\tmplength}
\settowidth{\labelsep}{X}
\addtolength{\leftmargin}{\labelsep}
\setlength{\labelwidth}{\tmplength}
}
\begin{flushleft}
\item[\textbf{Declaration}\hfill]
\begin{ttfamily}
procedure ThisShouldBeIncluded2;\end{ttfamily}


\end{flushleft}
\end{list}
\section{Constants}
\subsection*{ThisShouldBeTrue}
\begin{list}{}{
\settowidth{\tmplength}{\textbf{Description}}
\setlength{\itemindent}{0cm}
\setlength{\listparindent}{0cm}
\setlength{\leftmargin}{\evensidemargin}
\addtolength{\leftmargin}{\tmplength}
\settowidth{\labelsep}{X}
\addtolength{\leftmargin}{\labelsep}
\setlength{\labelwidth}{\tmplength}
}
\begin{flushleft}
\item[\textbf{Declaration}\hfill]
\begin{ttfamily}
ThisShouldBeTrue = true;\end{ttfamily}


\end{flushleft}
\end{list}
\subsection*{FourConst}
\begin{list}{}{
\settowidth{\tmplength}{\textbf{Description}}
\setlength{\itemindent}{0cm}
\setlength{\listparindent}{0cm}
\setlength{\leftmargin}{\evensidemargin}
\addtolength{\leftmargin}{\tmplength}
\settowidth{\labelsep}{X}
\addtolength{\leftmargin}{\labelsep}
\setlength{\labelwidth}{\tmplength}
}
\begin{flushleft}
\item[\textbf{Declaration}\hfill]
\begin{ttfamily}
FourConst =  (1 + 1) * (1 + 1);\end{ttfamily}


\end{flushleft}
\par
\item[\textbf{Description}]
Test of recursive macro expansion.

\end{list}
\subsection*{OneAndNotNothing}
\begin{list}{}{
\settowidth{\tmplength}{\textbf{Description}}
\setlength{\itemindent}{0cm}
\setlength{\listparindent}{0cm}
\setlength{\leftmargin}{\evensidemargin}
\addtolength{\leftmargin}{\tmplength}
\settowidth{\labelsep}{X}
\addtolength{\leftmargin}{\labelsep}
\setlength{\labelwidth}{\tmplength}
}
\begin{flushleft}
\item[\textbf{Declaration}\hfill]
\begin{ttfamily}
OneAndNotNothing = 1  + 1;\end{ttfamily}


\end{flushleft}
\par
\item[\textbf{Description}]
Test that symbol that is not a macro is something different than a macro that expands to nothing.

\end{list}
\subsection*{OnlyOne}
\begin{list}{}{
\settowidth{\tmplength}{\textbf{Description}}
\setlength{\itemindent}{0cm}
\setlength{\listparindent}{0cm}
\setlength{\leftmargin}{\evensidemargin}
\addtolength{\leftmargin}{\tmplength}
\settowidth{\labelsep}{X}
\addtolength{\leftmargin}{\labelsep}
\setlength{\labelwidth}{\tmplength}
}
\begin{flushleft}
\item[\textbf{Declaration}\hfill]
\begin{ttfamily}
OnlyOne = 1 ;\end{ttfamily}


\end{flushleft}
\end{list}
\end{document}
