\documentclass{report}
\usepackage{hyperref}
% WARNING: THIS SHOULD BE MODIFIED DEPENDING ON THE LETTER/A4 SIZE
\oddsidemargin 0cm
\evensidemargin 0cm
\marginparsep 0cm
\marginparwidth 0cm
\parindent 0cm
\textwidth 16.5cm

\ifpdf
  \usepackage[pdftex]{graphicx}
\else
  \usepackage[dvips]{graphicx}
\fi

\begin{document}
% special variable used for calculating some widths.
\newlength{\tmplength}
\chapter{Unit ok{\_}table{\_}nonlatex}
\section{Description}
Test of @table{-}related features that do not work in LaTeX, only in HTML.\hfill\vspace*{1ex}



Test that everything within @cell tag is OK:



\begin{tabular}{|l|l|}
\hline
\textbf{ Anything within a cell is OK, including lists: \begin{enumerate}
\setcounter{enumi}{0} \setcounter{enumii}{0} \setcounter{enumiii}{0} \setcounter{enumiv}{0} 
\item One
\setcounter{enumi}{1} \setcounter{enumii}{1} \setcounter{enumiii}{1} \setcounter{enumiv}{1} 
\item Two
\setcounter{enumi}{2} \setcounter{enumii}{2} \setcounter{enumiii}{2} \setcounter{enumiv}{2} 
\item Three
\end{enumerate}, paragraphs:

This is new paragraph.

Dashes: ---, --, {-}, {-}{-}. URLs: http://pasdoc.sourceforge.net/

And, last but not least, nested table:



\begin{tabular}{|l|l|}
\hline
\textbf{1} & \textbf{2} \\ \hline
\textbf{3} & \textbf{4} \\ \hline
\end{tabular}



} & \textbf{B} \\ \hline
C & D \\ \hline
\end{tabular}



Now the same example table as before, but now nested table and other nicies are within a normal row, instead of heading row.



\begin{tabular}{|l|l|}
\hline
\textbf{C} & \textbf{D} \\ \hline
 Anything within a cell is OK, including lists: \begin{enumerate}
\setcounter{enumi}{0} \setcounter{enumii}{0} \setcounter{enumiii}{0} \setcounter{enumiv}{0} 
\item One
\setcounter{enumi}{1} \setcounter{enumii}{1} \setcounter{enumiii}{1} \setcounter{enumiv}{1} 
\item Two
\setcounter{enumi}{2} \setcounter{enumii}{2} \setcounter{enumiii}{2} \setcounter{enumiv}{2} 
\item Three
\end{enumerate}, paragraphs:

This is new paragraph.

Dashes: ---, --, {-}, {-}{-}. URLs: http://pasdoc.sourceforge.net/

And, last but not least, nested table:



\begin{tabular}{|l|l|}
\hline
\textbf{1} & \textbf{2} \\ \hline
\textbf{3} & \textbf{4} \\ \hline
\end{tabular}



 & B \\ \hline
\end{tabular}


\end{document}
